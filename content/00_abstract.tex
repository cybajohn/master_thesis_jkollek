%abstract
\section*{Abstract}

This work investigates the contribution of neutrinos at the positions of $\num{72}$ gamma-ray-followup (GFU) neutrino alerts with $\num{9}$ years of IceCube data.
Two analyses are presented, a time-integrated stacking search with all sources and individual time-dependent searches at $\num{10}$ selected source positions.
Following a conservative approach, the simulation data is cleaned in advance from GFU-like events.
The time-dependent search fixes the emission centre and allows for uniformly distributed emission intervals of up to $\num{200}$ days based on the analysis time window of TXS 0506+056 \cite{txs}.
The time-integrated search yields a sensitivity of dudu and a $\num{5}\sigma$ discovery potential of dudu.
The best result in the time-integrated search yields (name here) with a sensitivity of dudu and a $\num{5}\sigma$ discovery potential of dudu.
Meaning...
In addition, the problem of time-dependent stacking is discussed using the example of specially modified software and an example of the calculation is given.

\section*{Kurzfassung}

Diese Arbeit untersucht den Beitrag von Neutrinos an den Positionen von $\num{72}$ gamma-ray-followup (GFU) Neutrino Alerts mit $\num{9}$ Jahren von IceCube Daten.
Es werden zwei Analysen presentiert, eine zeitintegrierte Stacking Suche mit allen Quellen und individuelle zeitabhängige Suchen an $\num{10}$ ausgewählten Quellpositionen.
Einem konservativen Ansatz folgend werden die Simulationsdaten im voraus von GFU-artigen Ereignissen bereinigt.
Die zeitliche Emission bei der zeitabhängigen Suche fixiert den Emissionsmittelpunkt und erlaubt gleichverteilte Emissionsintervalle von bis zu $\num{200}$ Tagen angelehnt an dem Analysezeitfenster von TXS 0506+056 \cite{txs}.
Die zeitintegrierte Suche liefert eine Sensitivität von dudu und ein $\num{5}\sigma$ Discovery Potential von dudu.
Das beste Ergebnis bei der zeitabhängigen Suche liefert (hier Name) mit einer Sensitivität von dudu und einem $\num{5}\sigma$ Discovery Potential von dudu.
das bedeutet...
Zusätzlich wird auf der Problematik eines zeitabhängigen Stackings am Beispiel eigens modifizierter Software eingegangen und ein Beispiel der Berechnung gegeben.
