%abstract
\section*{Abstract}

This work investigates the contribution of neutrinos at the positions of $\num{72}$ gamma-ray-followup (GFU) neutrino alerts with $\num{9}$ years of IceCube data.
Two analyses are presented, a time-integrated stacking search with all sources and individual time-dependent searches at $\num{10}$ selected source positions.
Following a conservative approach, the simulation data is cleaned in advance from GFU-like events.
The time-dependent search fixes the emission centre and allows for uniformly distributed emission intervals of up to $\num{200}$ days.
This is based on the analysis time windows of a time-dependent point source search for neutrinos from the direction of the active galactic nucleus TXS 0506+056 \cite{_txs}.
The time-integrated search yields a sensitivity of $\SI{1.68e-14}{\tera\electronvolt\tothe{-1}\centi\meter\tothe{-1}\second\tothe{-1}}$ and a $\num{5}\sigma$ discovery potential of $\SI{6.81e-14}{\tera\electronvolt\tothe{-1}\centi\meter\tothe{-1}\second\tothe{-1}}$.
The best result in the time-dependent search yields the GFU-gold event from 03.31.2016 with a sensitivity of $\SI{4.63e-2}{\giga\electronvolt\centi\meter\tothe{-2}}$ and a $\num{5}\sigma$ discovery potential of $\SI{1.32e-1}{\giga\electronvolt\centi\meter\tothe{-2}}$.
These results are in range with comparable analyses.
In addition, methods for a time-dependent stacking analysis have been developed and its challenges are discussed.

\section*{Kurzfassung}

Diese Arbeit untersucht den Beitrag von Neutrinos an den Positionen von $\num{72}$ gamma-ray-followup (GFU) Neutrino Alerts mit $\num{9}$ Jahren von IceCube Daten.
Es werden zwei Analysen präsentiert, eine zeitintegrierte Stacking-Suche mit allen Quellen und individuelle zeitabhängige Suchen an $\num{10}$ ausgewählten Quellpositionen.
Einem konservativen Ansatz folgend werden die Simulationsdaten im Voraus von GFU-artigen Ereignissen bereinigt.
Die zeitliche Emission bei der zeitabhängigen Suche fixiert den Emissionsmittelpunkt und erlaubt gleichverteilte Emissionsintervalle von bis zu $\num{200}$ Tagen.
Dies ist angelehnt an die Analysezeitfenster von einer zeitabhängigen Punktquellensuche nach Neutrinos aus dem aktiven Galaxienkern TXS 0506+056 \cite{_txs}.
Die zeitintegrierte Suche liefert eine Sensitivität von $\SI{1.68e-14}{\tera\electronvolt\tothe{-1}\centi\meter\tothe{-1}\second\tothe{-1}}$ und ein $\num{5}\sigma$ Discovery Potential von $\SI{6.81e-14}{\tera\electronvolt\tothe{-1}\centi\meter\tothe{-1}\second\tothe{-1}}$.
Das beste Ergebnis bei der zeitabhängigen Suche liefert das GFU-gold Events vom 31.03.2016 mit einer Sensitivität von $\SI{4.63e-2}{\giga\electronvolt\centi\meter\tothe{-2}}$ und einem $\num{5}\sigma$ Discovery Potential von $\SI{1.32e-1}{\giga\electronvolt\centi\meter\tothe{-2}}$.
Diese Ergebnisse liegen in dem Bereich vergleichbarer Analysen.
Darüber hinaus wurden Methoden für eine zeitabhängige Stackinganalyse entwickelt, und die damit verbundenen Herausforderungen werden diskutiert.
