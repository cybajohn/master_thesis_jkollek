\chapter{Astroparticle physics} \label{sec:astro}

Cosmic particles have always offered an insight into the mechanisms of cosmic objects.
For example, pulsars and quasars have been discovered through radio astronomy.
Today, information is obtained from all kinds of cosmic particles, including weakly interacting neutrinos.
The production of neutrinos is due to charged particles that have been accelerated by a type of shock front acceleration in magnetic fields.
These are mainly charged hadronic particles such as the proton, which, in the framework of the standard model of particle physics, produces a pion mainly through interaction with other protons (protohadronic) or photons (photohadronic).
In the course of the pion's decay, neutrinos are ultimately created \cite{pdg},
\begin{align}
  &p+p \rightarrow p+n+\pi^+ \quad \text{or} \quad p\gamma \rightarrow n+\pi^+,\\
  &\pi^+ \rightarrow \mu^++\nu_\mu \rightarrow e^++\nu_e+\bar{\nu}_\mu+\nu_\mu.
\end{align}
In contrast to charged cosmic particles, which can be deflected in magnetic fields on their way through the cosmic medium, neutrinos and gamma rays provide a direct path to the source.
In addition, only neutrinos provide unequivocal evidence of accelerated protons, since high-energy gamma rays can also be produced by inverse Compton-scattering \cite{spiering}.


\section{Possible neutrino sources and their properties}

Various astrophysical objects are suitable potential sources for neutrinos.
The assignment of neutrinos to these objects could provide information on the theoretical description of these objects and their physical properties.
Two examples are described below, blazars and starburstgalaxies.

\subsection{Blazars}
- type of AGN \\
- inner jets cannot explain sub-PeV neutrino events cause of low energy cutoff \\
- needs more complicated models \\
\cite{blazar}

\subsection{Starburstgalaxies}
%- starburstgalaxies are galaxies with star formation at extreme levels\\
%- example: Arp 220\\
%- consists of two nuclei\\
%- star formation rate of > 100 sun mass per year\\
%- "Arp 220 is a hadronic cosmic ray calorimeter where all the power in %cosmic rays is absorbed within the nuclear starburst zones"\\
%- up to a hundred percent proton calorimeters\\
%- pp interaction -> pions -> 1.neutrino + charged lepton\\
%- flux of 1. neutrino is about $\num{6e-%12}\si{\giga\electronvolt\centi\meter\tothe{-2}\second\tothe{-1}}$ at %$\SI{0.1}{\peta\electronvolt}$\\
%\cite{starburst}
%- neutrinos from pp have a typical energy of about %$\SI{2}{\peta\electronvolt}$
%\cite{starburst2}

Apart from blazars, starburstgalaxies are candidates for the production of neutrinos of energies up to about $\SI{2}{\peta\electronvolt}$ \cite{starburst2}.
These are galaxies in which stars form at extreme levels.
One example is Arp 220, an object consisting of two nuclei surrounded by dense molecular gas.
The gas turns the object almost completely into a cosmic ray calorimeter.
Under certain model assumptions, the $pp$-interactions provide a neutrino flux for the first part of the interaction of about $\num{6e-12}\si{\giga\electronvolt\centi\meter\tothe{-2}\second\tothe{-1}}$ at $\SI{0.1}{\peta\electronvolt}$ \cite{starburst}.
