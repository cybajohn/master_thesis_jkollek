%sources
\chapter{Event Topology and Datasets}
%general: cascades and tracks\\
what are the best sources and why\\
probability of being of astrophysical origin\\
History of EHE, HESE to bronze, gold alerts Thomas Kintschers gfu\\
what is a gfu\\
hängt ab von zen, dec, track\_len, bdt\_up (upgoing oder nicht) oder bei donwgoing: truncated energy, qtot (deponierte Ladung)\\
\_zen\_cut = np.radians(82.)\\
up: pass\_gold[m]   = pass\_precut[m] \& (qtot[m] > 10.\*\*np.polyval(\_coeff2, np.sin(dec[m])))\\
\_coeff2 = [  -4.06580, -10.60906,  -9.61048,   3.01219 ]\\
down: pass\_gold[m]   \= pass\_precut[m] \& (logTruncated[m] > \_cut2)\\
\_cut2   = 5.14\\

table of sources, skymaps\\

Historically events were categorized between high energy starting events (HESE) and extremely high energy events (EHE).
Starting events are as the name suggests events starting inside the detector volume (interaction vertex)

\cite{Aartsen_2017}

\section{Gamma-Ray-Followup}
