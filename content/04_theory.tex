%theory
\chapter{Point Source Searches in IceCube}

mention programs like skylab sky llh and yeah csky duh

maybe mention other methods than likelihood

Sensitivity and discovery potential correspond in principle to the number of signal events that must be injected, so that the resulting test statistic is $\SI{90}{\percent}$ above the median of the background statistic or $\SI{50}{\percent}$ above $\num{5}\sigma$ of the latter respectively.
These conditions can be seen schematically in figure \ref{fig:sens_disc_schem}.
Let the number of sufficient signal events then be $N_\text{sig}$, the sensitivity $sens$ and discovery potential $disc p$ then are
\begin{align}
  sens &= \frac{N_\text{sig,sens}}{A_\text{det}}E_0, \\
  disc p &= \frac{N_\text{sig,disc p}}{A_\text{det}}E_0,
\end{align}
with the total signal acceptance of the detector $A_\text{det}$ and the reference energy $E_0$.
The time-dependent sensitivity and discovery potential are given in a fluence rather than a flux by multiplying with $E_0^2$, these results can then be compared to other analyses.

\section{Likelihood in csky}

\begin{align}
  \CL(n_s,\Vmu)
  = \prod_i \Gb{
    \frac{n_s}{N} \CS(\Vx_i|\Vmu)
    + \Gp{1 - \frac{n_s}{N}} \CB(\Vx_i)
  }.
  \label{eq:lh}
\end{align}
