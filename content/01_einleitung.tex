%introduction
\chapter{Introduction}
Talk about txs, short meaning if sources were found, thorbens analysis, my motivation, short summary of contents (talk about common thread)\\

what differs in this pss from others\\
short introduction to chapters\\

Chapter \ref{sec:astro} describes the basic physical processes that generate neutrinos and possible objects that can be their source.
Chapter \ref{sec:icecube} outlines the IceCube Observatory and how neutrinos are detected, as well as the appearance and classification of the detector signatures.
The underlying methodology of the point source search is provided in the theory-oriented Chapter \ref{sec:theory}.
Chapter \ref{sec:events_data} describes the data sets used and processed, as well as the source catalogue.
The next two chapters,\ref{sec:csky_time_int} and \ref{sec:csky_time_dep} , contain the time-integrated and the time-dependent analyses.
This is followed by chapter \ref{sec:tdepps}, which deals with the problem of time-dependent stacking and the attempt to solve this problem.
Finally, chapter \ref{sec:outlook} provides a summary and a discussion of the results as well as ideas for further evaluations.
Additional material for further understanding and reproducibility of the thesis is listed in the appendix \ref{sec:appendix}.
