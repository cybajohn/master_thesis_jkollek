%introduction
\chapter{Introduction}
Talk about txs, short meaning if sources were found, thorbens analysis, my motivation, short summary of contents (talk about common thread)\\

name some source catalogues\\
blazar stacking\\
martina karl\\
what differs in this pss from others: csky, removing gfu-gold mc events, method of time dep search\\
own attempt at software\\
IceCube sensitivity and what to expect from analysis results\\
short introduction to chapters\\

blazarstacking2022:\\
- multi-messenger astronomy: high-energy photons are likely to get absorbed and reemitted at lower energies\\
- thus: time-integrated stacking of MeV-detected blazars using 10 years of IceCube data\\
- upper flux limit is $E^2\phi_{\SI{1}{\tera\electronvolt}}=\SI{1.64e-12}{\tera\electronvolt\centi\meter\tothe{-2}\second\tothe{-1}}$\\
- \cite{blazar_stacking_2020}

In general, point source searches are divided into two types, allsky scans and catalogue searches.
Allsky scans divide the sky into declination bands and thus examine the whole sky for a neutrino excess.
(here allsky paper).
Catalogue searches, on the other hand, are motivated by theoretical assumptions, such as that certain astrophysical objects should emit neutrinos based on theoretical models.
Popular candidates for possible neutrino sources are active galactic nuclei.
For example, a high-energy neutrino was assigned to the blazar TXS 0506+056 in 2018 \cite{txs}.
This was the starting signal for catalogue searches with blazars, in which a low-energy neutrino excess is sought.
Stacking the source positions enables a higher sensitivity.
The blazar stackings performed showed no significant excess with an upper limit of dudu for dudu-stacked blazars.
A more recent search \cite{blazar_stacking_2020} uses blazars of low-energy photons, motivated by the theory that high-energy photons related to neutrino production are absorbed and reemitted at lower energies.
This search with $\num{137}$ $\si{\mega\electronvolt}$-detected blazars and $\num{11}$ years of IceCube data yields an upper limit of $E^2\phi_{\SI{1}{\tera\electronvolt}}=\SI{1.64e-12}{\tera\electronvolt\centi\meter\tothe{-2}\second\tothe{-1}}$.
In addition to the approach of examining the entire data set for neutrino emission in the time-integrated searches just mentioned, there is also the possibility of examining smaller time intervals for neutrino flares in a so-called time-dependent search.
(here martina karl).

Chapter \ref{sec:astro} describes the basic physical processes that generate neutrinos and possible objects that can be their source.
Chapter \ref{sec:icecube} outlines the IceCube Observatory and how neutrinos are detected, as well as the appearance and classification of the detector signatures.
The underlying methodology of the point source search is provided in the theory-oriented Chapter \ref{sec:theory}.
Chapter \ref{sec:events_data} describes the data sets used and processed, as well as the source catalogue.
The next two chapters,\ref{sec:csky_time_int} and \ref{sec:csky_time_dep} , contain the time-integrated and the time-dependent analyses.
This is followed by chapter \ref{sec:tdepps}, which deals with the problem of time-dependent stacking and the attempt to solve this problem.
Finally, chapter \ref{sec:outlook} provides a summary and a discussion of the results as well as ideas for further evaluations.
Additional material for further understanding and reproducibility of the thesis is listed in the appendix \ref{sec:appendix}.
