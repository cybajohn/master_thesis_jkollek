%both results
\chapter{Conclusion and Outlook} \label{sec:outlook}

It is possible to compare these results with the diffuse astrophysical muon neutrino flux as is done, for example, in \cite{thorben}.
Using the flux model from $\num{10}$ years of IceCube data \cite{flux_2},
\begin{align}
  \Phi^{\nu_\mu + \overline{\nu}_\mu} &= 1.44\cdot\left(\frac{E_\nu}{\SI{100}{\tera\electronvolt}}\right)^{-2.28}\cdot\SI{e-18}{\giga\electronvolt\tothe{-1}\centi\meter\tothe{-1}\second\tothe{-1}\steradian\tothe{-1}},
\end{align}
a fluence upper limit can be generated.
It is assumed that the entire flux comes from these $\num{10}$ sources in the form of a burst emission.
The burst rate $R$ can be calculated with the total lifetime $T$ taken from table \ref{tab:data}
\begin{align}
  R &= \frac{N_\text{srcs}}{4\pi\cdot T}
\end{align}
and the fluence at a refernece energy of $\SI{100}{\tera\electronvolt}$ is about
\begin{align}
  E^2_\nu\Phi_0^{\nu_\mu + \overline{\nu}_\mu} &\approx \SI{1.44e-8}{\giga\electronvolt\centi\meter\tothe{-2}\second\tothe{-1}\steradian\tothe{-1}}.
\end{align}
The per-burst fluence with the burst time window $\Delta T$ canceling out then becomes
\begin{align}
  E^2_\nu\Phi_\text{GFU}^\text{diff} &= \frac{\Delta T \cdot E^2_\nu\Phi_0^{\nu_\mu + \overline{\nu}_\mu} \cdot 4\pi}{\Delta T \cdot R \cdot 4\pi},\\
  &\approx \SI{4.98}{\giga\electronvolt\centi\meter\tothe{-2}}.
\end{align}


are the results measurable?\\
time-int sens: $\SI{1.68e-14}{\tera\electronvolt\tothe{-1}\centi\meter\tothe{-1}\second\tothe{-1}}$ per source\\
time-int disc $\num{5}\sigma$: $\SI{6.81e-14}{\tera\electronvolt\tothe{-1}\centi\meter\tothe{-1}\second\tothe{-1}}$ per source\\
one magnitude under all-sky\\
time-int total flux of $\SI{4.11}{\giga\electronvolt\centi\meter\tothe{-2}\second\tothe{-1}}$ calculated as thorben at $\SI{100}{\tera\electronvolt}$ reference\\
time-dep sens, best: $\SI{4.63e-2}{\giga\electronvolt\centi\meter\tothe{-2}}$ at $\SI{100}{\tera\electronvolt}$ reference\\
time-dep disc $\num{5}\sigma$, best: $\SI{1.32e-1}{\giga\electronvolt\centi\meter\tothe{-2}}$ at $\SI{100}{\tera\electronvolt}$ reference\\
time-dep weighted per burst fluence of $\SI{34.91}{\giga\electronvolt\centi\meter\tothe{-2}}$ calculated as thorben at $\SI{100}{\tera\electronvolt}$ reference\\
martina karl: disc $\SI{2.7e-2}{\giga\electronvolt\centi\meter\tothe{-2}}$ with $\num{3}\sigma$ but \textbf{what reference energy?}\\
compare both analyses, \textbf{but how?}\\
compare with other analyses\\

what could have been done better, whats weird:\\
- mc gfu gold events about times 2: 21002 21220 are weird\\
- source 37 has few trials\\
- more trials; timewindow distribution could be smoother\\
- maybe theres a background event which is an outlier that gets badly scrambled\\
- post trials should be done or all sources\\
- different gammas\\
- take gamma of gfu gold events\\
- spatial errors of sources could be accounted for\\
- take newest version of ps\_tracks\\
why no unblinding?\\
- takes too long\\

All in all, this work provides a good basis for getting started with point source searches using \texttt{csky} and can be used for further analyses by making minor changes.
For example, the data set can be enlarged or additional or different source locations can be investigated.
