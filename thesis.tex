%%%%%%%%%%%%%%%%%%%%%%%%%%%%%%%%%%%%%%%%%%%%%%%%%%%%%%%%%%%%%%%%%%%%%%%%%%%%%%%%
%%%%%%%%%%%%%%%%%%   Vorlage für eine Abschlussarbeit   %%%%%%%%%%%%%%%%%%%%%%%%
%%%%%%%%%%%%%%%%%%%%%%%%%%%%%%%%%%%%%%%%%%%%%%%%%%%%%%%%%%%%%%%%%%%%%%%%%%%%%%%%

% Erstellt von Maximilian Nöthe, <maximilian.noethe@tu-dortmund.de>
% ausgelegt für lualatex und Biblatex mit biber

% Kompilieren mit
% latexmk --lualatex --output-directory=build thesis.tex
% oder einfach mit:
% make

\documentclass[
  tucolor,       % remove for less green,
  BCOR=12mm,     % 12mm binding corrections, adjust to fit your binding
  parskip=half,  % new paragraphs start with half line vertical space
  open=any,      % chapters start on both odd and even pages
  cleardoublepage=plain,  % no header/footer on blank pages
]{tudothesis}


% Warning, if another latex run is needed
\usepackage[aux]{rerunfilecheck}

% just list chapters and sections in the toc, not subsections or smaller
\setcounter{tocdepth}{1}

%------------------------------------------------------------------------------
%------------------------------ Fonts, Unicode, Language ----------------------
%------------------------------------------------------------------------------
\usepackage{fontspec}
\defaultfontfeatures{Ligatures=TeX}  % -- becomes en-dash etc.

% english language
\usepackage{polyglossia}
\setdefaultlanguage{english}

% for english abstract and english titles in the toc
\setotherlanguages{english}

% intelligent quotation marks, language and nesting sensitive
\usepackage[autostyle]{csquotes}

% microtypographical features, makes the text look nicer on the small scale
\usepackage{microtype}

%------------------------------------------------------------------------------
%------------------------ Math Packages and settings --------------------------
%------------------------------------------------------------------------------

\usepackage{amsmath}
\usepackage{amssymb}
\usepackage{mathtools}

% Enable Unicode-Math and follow the ISO-Standards for typesetting math
\usepackage[
  math-style=ISO,
  bold-style=ISO,
  sans-style=italic,
  nabla=upright,
  partial=upright,
]{unicode-math}
\setmathfont{Latin Modern Math}

% nice, small fracs for the text with \sfrac{}{}
\usepackage{xfrac}


%------------------------------------------------------------------------------
%---------------------------- Numbers and Units -------------------------------
%------------------------------------------------------------------------------

\usepackage[
  locale=UK,
  separate-uncertainty=true,
  per-mode=symbol-or-fraction,
]{siunitx}
\sisetup{math-micro=\text{µ},text-micro=µ}

%------------------------------------------------------------------------------
%-------------------------------- tables  -------------------------------------
%------------------------------------------------------------------------------

\usepackage{booktabs}       % \toprule, \midrule, \bottomrule, etc

%------------------------------------------------------------------------------
%-------------------------------- graphics -------------------------------------
%------------------------------------------------------------------------------

\usepackage{graphicx}
\usepackage{draftfigure}
% currently broken
% \usepackage{grffile}

% allow figures to be placed in the running text by default:
\usepackage{scrhack}
\usepackage{float}
\floatplacement{figure}{htbp}
\floatplacement{table}{htbp}

% keep figures and tables in the section
\usepackage[section, below]{placeins}


%------------------------------------------------------------------------------
%---------------------- customize list environments ---------------------------
%------------------------------------------------------------------------------

\usepackage{enumitem}

%------------------------------------------------------------------------------
%------------------------------ Bibliographie ---------------------------------
%------------------------------------------------------------------------------

\usepackage[
  backend=biber,   % use modern biber backend
  sorting=none,    % references are in appearance order
  autolang=hyphen, % load hyphenation rules for if language of bibentry is not
                   % english, has to be loaded with \setotherlanguages
                   % in the references.bib use langid={en} for english sources
]{biblatex}
\addbibresource{references.bib}  % the bib file to use
\DefineBibliographyStrings{english}{andothers = {{et\,al\adddot}}}  % replace u.a. with et al.


% Last packages, do not change order or insert new packages after these ones
\usepackage[pdfusetitle, unicode, linkbordercolor=tugreen]{hyperref}
\usepackage{bookmark}
\usepackage[shortcuts]{extdash}

%------------------------------------------------------------------------------
%-------------------------   Some commands ------------------------------------
%------------------------------------------------------------------------------

% tiks ticks ist kein zeichenprogramm :)
\usepackage{tikz}
\usetikzlibrary{calc}
\usetikzlibrary{arrows,patterns}
\usetikzlibrary{plotmarks}
\usetikzlibrary{decorations.pathmorphing}

\tikzset{snake it/.style={decorate, decoration=snake}}

\usepackage{pgfplots}
\pgfmathdeclarefunction{gauss}{3}{%
  \pgfmathparse{#3*1/(#2*sqrt(2*pi))*exp(-((x-#1)^2)/(2*#2^2))}%
}

% some math stuff to make things easier, hate the left right stuff....
\newcommand{\CL}{\ensuremath{\mathcal{L}}}
\newcommand{\CS}{\ensuremath{\mathcal{S}}}
\newcommand{\CB}{\ensuremath{\mathcal{B}}}
\newcommand{\CP}{\ensuremath{\mathcal{P}}}
\newcommand{\Vx}{\ensuremath{\vec{x}}}
\newcommand{\Vmu}{\ensuremath{\vec{\mu}}}

\newcommand{\Gp}[1]{\left( #1 \right)}
\newcommand{\Gb}[1]{\left[ #1 \right]}
\newcommand{\GB}[1]{\left\{ #1 \right\}}

%------------------------------------------------------------------------------
%-------------------------    Angaben zur Arbeit   ----------------------------
%------------------------------------------------------------------------------

\author{Johannes Kollek}
\title{Point Source Search for a Neutrino Contribution at Neutrino Alert Positions}
\date{2023}
\birthplace{Kamp-Lintfort}
\chair{Lehrstuhl für Experimentelle Physik V}
\division{Fakultät Physik}
\thesisclass{Master of Science}
\submissiondate{tba}
\firstcorrector{Prof.~Dr.~Erstgutachter}
\secondcorrector{Prof.~Dr.~Zweitgutachter}

% tu logo on top of the titlepage
\titlehead{\includegraphics[height=1.5cm]{logos/tu-logo.pdf}}

\begin{document}
\frontmatter
%\input{content/hints.tex}
\maketitle

% Gutachterseite
\makecorrectorpage

% hier beginnt der Vorspann, nummeriert in römischen Zahlen
%abstract
\section*{Abstract}

This work investigates the contribution of neutrinos at the positions of $\num{72}$ gamma-ray-followup (GFU) neutrino alerts with $\num{9}$ years of IceCube data.
Two analyses are presented, a time-integrated stacking search with all sources and individual time-dependent searches at $\num{10}$ selected source positions.
Following a conservative approach, the simulation data is cleaned in advance from GFU-like events.
The time-dependent search fixes the emission centre and allows for uniformly distributed emission intervals of up to $\num{200}$ days.
This is based on the analysis time windows of a time-dependent point source search for neutrinos from the direction of the active galactic nucleus TXS 0506+056 \cite{_txs}.
The time-integrated search yields a sensitivity of $\SI{1.68e-14}{\tera\electronvolt\tothe{-1}\centi\meter\tothe{-1}\second\tothe{-1}}$ and a $\num{5}\sigma$ discovery potential of $\SI{6.81e-14}{\tera\electronvolt\tothe{-1}\centi\meter\tothe{-1}\second\tothe{-1}}$.
The best result in the time-dependent search yields the GFU-gold event from 03.31.2016 with a sensitivity of $\SI{4.63e-2}{\giga\electronvolt\centi\meter\tothe{-2}}$ and a $\num{5}\sigma$ discovery potential of $\SI{1.32e-1}{\giga\electronvolt\centi\meter\tothe{-2}}$.
These results are in range with comparable analyses.
In addition, methods for a time-dependent stacking analysis have been developed and its challenges are discussed.

\section*{Kurzfassung}

Diese Arbeit untersucht den Beitrag von Neutrinos an den Positionen von $\num{72}$ gamma-ray-followup (GFU) Neutrino Alerts mit $\num{9}$ Jahren von IceCube Daten.
Es werden zwei Analysen präsentiert, eine zeitintegrierte Stacking-Suche mit allen Quellen und individuelle zeitabhängige Suchen an $\num{10}$ ausgewählten Quellpositionen.
Einem konservativen Ansatz folgend werden die Simulationsdaten im Voraus von GFU-artigen Ereignissen bereinigt.
Die zeitliche Emission bei der zeitabhängigen Suche fixiert den Emissionsmittelpunkt und erlaubt gleichverteilte Emissionsintervalle von bis zu $\num{200}$ Tagen.
Dies ist angelehnt an die Analysezeitfenster von einer zeitabhängigen Punktquellensuche nach Neutrinos aus dem aktiven Galaxienkern TXS 0506+056 \cite{_txs}.
Die zeitintegrierte Suche liefert eine Sensitivität von $\SI{1.68e-14}{\tera\electronvolt\tothe{-1}\centi\meter\tothe{-1}\second\tothe{-1}}$ und ein $\num{5}\sigma$ Discovery Potential von $\SI{6.81e-14}{\tera\electronvolt\tothe{-1}\centi\meter\tothe{-1}\second\tothe{-1}}$.
Das beste Ergebnis bei der zeitabhängigen Suche liefert das GFU-gold Events vom 31.03.2016 mit einer Sensitivität von $\SI{4.63e-2}{\giga\electronvolt\centi\meter\tothe{-2}}$ und einem $\num{5}\sigma$ Discovery Potential von $\SI{1.32e-1}{\giga\electronvolt\centi\meter\tothe{-2}}$.
Diese Ergebnisse liegen in dem Bereich vergleichbarer Analysen.
Darüber hinaus wurden Methoden für eine zeitabhängige Stackinganalyse entwickelt, und die damit verbundenen Herausforderungen werden diskutiert.

\tableofcontents

\mainmatter
% Hier beginnt der Inhalt mit Seite 1 in arabischen Ziffern
%introduction
\chapter{Introduction}
Talk about txs, short meaning if sources were found, thorbens analysis, my motivation, short summary of contents (talk about common thread)\\

what differs in this pss from others\\
short introduction to chapters\\

\chapter{Astroparticle physics} \label{sec:astro}

general how neutrinos are created pp pgamma
what accelerates the charged particles


\section{Neutrino Sources and their Properties}

AGN: hadronic SED stuff\\
SBG: energies n stuff\\
other sources?

\chapter{IceCube Neutrino Observatory} \label{sec:icecube}

Because of the small cross section of neutrinos, a huge detection volume is needed to detect these particles in a statistically sufficient quantity \cite{cross_n}.
Since the detection of neutrinos is based on the Cherenkov radiation of secondary particles, a detection medium is needed that is transparent to radiation in this range.
With regard to these two problems, the kilometer-thick ice at the south pole is a good choice.
The resulting IceCube neutrino observatory is the world's largest particle detector.
It is almost exactly located at the southernmost point of the earth and shown schematically in figure \ref{fig:icecube}.

\begin{figure}
    \centering
    \includegraphics[width=\linewidth]{Plots/01_7_icecube/IceCube-Array.jpg}
    \caption{Schematic representation of the IceCube detector \cite{icecube_website}.}
    \label{fig:icecube}
\end{figure}

Its construction started $\num{2004}$ and took $\num{7}$ years to its completion in $\num{2010}$.
The detector volume of IceCube covers about $\SI{1}{\kilo\meter\tothe{3}}$ of clear antarctic ice.
$\num{86}$ holes were drilled into the ice to let down cables with $\num{60}$ digital optical modules (DOMs) each.
The detecion volume, or the position of the first DOM respectively, starts at a depth of about $\SI{1450}{\meter}$ and ends in a depth of about $\SI{2450}{\meter}$.
The area known as the DeepCore in the middle of the detector features a denser collection of DOMs.
IceCube detects about $\num{275}$ million cosmic rays and $\num{275}$ atmospheric neutrinos every day \cite{icecube_website}.

\section{Detection Principle}

When neutrinos interact with a medium, charged secondary particles are produced.
this occurs through a neutral current (NC) via a $Z^0$ boson or through a charged current (CC) via a $W$ boson
\begin{equation}
  \nu_l + N \rightarrow Z^0 \rightarrow \nu^{\prime}_l + X \text{ and }  \nu_l + N \rightarrow W \rightarrow l + X, \label{eq:nc_cc_int}
\end{equation}
with $N$ and $X$ being nuclei before and after the interaction \cite{Ahlers_2018}.
If the charged secondary particles (the charged leptons $l$ in \eqref{eq:nc_cc_int}) now move through a medium at a speed faster than the speed of light in the medium, the charged secondary particles polarise the atoms along the direction of flight for a short time.
The resulting light is called Cherenkov light and is emitted at an angle of
\begin{equation}
  \Theta = \arccos{\left(\frac{1}{\beta n}\right)}, \label{eq:cherenkov}
\end{equation}
with the relativistic velocity $\beta=v/c$ of the particle and the refractive index of the medium $n$ \cite{PhysRev52378} \cite{spiering}.
A schematic representation of the Cherenkov effect is given in Figure \ref{fig:cherenkov}.
So-called tracks are created by charged current interactions and cascades with both charged current and neutral current interactions.
\begin{figure}
  \centering
  \scalebox{0.5}{
  \begin{tikzpicture}[scale=1,
    >=stealth',
    pos=.8,
    photon/.style={decorate,decoration={snake,post length=1mm}},
    declare function={
    x1 = (sqrt(25-5*sin(45)));
    x2 = 10;
    y1 = 5*sin(45);
    y2 = 0;
    m = (5*sqrt(2)/(-20+sqrt(100-10*sqrt(2))));
    b =  (-50*sqrt(2)/(-20+sqrt(100-10*sqrt(2))));
    o = 0.5 * sqrt(25-5*sin(45));
    p = 0.5*5*sin(45);
    }
    ]
    %\draw[thick] (-3,0) -- (0,0);
    %triangle
    \draw[thick] (0,0) -- (10,0) -- (sqrt{25-5*sin{45}},5*sin{45}) -- cycle;
    \draw[thick] (0,0) -- (10,0) -- (sqrt{25-5*sin{45}},-5*sin{45});
    \draw[thick] ([shift=(0:1cm)]0,0) arc (0:67:1cm);
    \draw[thick] ([shift=(-113:1cm)]sqrt{25-5*sin{45}},5*sin{45}) arc (-113:-23:1cm);
    \node at (0.5,0.4) {$\Theta$};
    \node[mark size=1pt] at (sqrt{25-5*sin{45}} + 0.25,5*sin{45} - 0.53) {\pgfuseplotmark{*}};
    \node[scale=2] at (0.2,0.5*5*sin{45}) {$\frac{c}{n}$};
    \node[below,scale=2] at (4.5,-0.2) {$v$};
    %circles
    \draw[teal, thick] (0,0) circle (3.82);
    \draw[teal, thick] (4,0) circle (2.28);
    \draw[teal, thick] (6.5,0) circle (1.32);
    \draw[teal, thick] (8,0) circle (0.75);
    %particles
    \draw[teal,thick, ->] (10,0) -- (11,0);
    \node[mark size=2pt,teal] at (10,0) {\pgfuseplotmark{*}};
    \node[teal,scale=2] at (10,1) {$l$};
    \draw[thick,blue,photon, ->] (sqrt{25-5*sin{45}},5*sin{45}) -- node[below right,scale=2] {$\gamma$} (sqrt{25-5*sin{45}}+0.6,5*sin{45}+1.3);
    \draw[thick,blue,photon, ->] (4.9,2.1) -- (4.9+0.6,2.1+1.3);
    \draw[thick,blue,photon, ->] (7,1.25) -- (7+0.6,1.25+1.3);
    \draw[thick,blue,photon, ->] (8.3,0.7) -- (8.3+0.6,0.7+1.3);
    \draw[thick,blue,photon, ->] (sqrt{25-5*sin{45}},-5*sin{45}) -- (sqrt{25-5*sin{45}}+0.6,-5*sin{45}-1.3);
    \draw[thick,blue,photon, ->] (4.9,-2.1) -- (4.9+0.6,-2.1-1.3);
    \draw[thick,blue,photon, ->] (7,-1.25) -- (7+0.6,-1.25-1.3);
    \draw[thick,blue,photon, ->] (8.3,-0.7) -- (8.3+0.6,-0.7-1.3);
    %\draw[thick,blue,->] (sqrt{25-5*sin{45}},5*sin{45}) -- (sqrt{25-5*sin{45}}+0.5*sqrt{25-5*sin{45}},5*sin{45}+0.5*5*sin{45});
    %\draw[thick,blue,->] (sqrt{25-5*sin{45}},m*x1+b) -- (sqrt{25-5*sin{45}}+0.5*sqrt{25-5*sin{45}},m*x1+b+0.5*5*sin{45});%(5,y*(5-sqrt{25-5*sin{45}})) -- (0,0);%(5+0.2*sqrt{25-5*sin{45}},y*(5-sqrt{25-5*sin{45}}+0.2*5*sin{45}));
  \end{tikzpicture}}
  \caption{Schematic visualizing the cherenkov radiation $\gamma$ emitted by a charged lepton $l$ traveling with velocity $v>c/n$ trough a medium with a refractive index $n$ polarizing the medium on its way. The resulting cone of cherenkov radiation is visually comparable to the supersonic cone of an aeroplane.}
  \label{fig:cherenkov}
\end{figure}

\section{Tracks and Cascades}

To perform a point source search, two properties of the detector signatures of the muons are of particular importance: the direction and the energy.
It is therefore important to distinguish between two types of events, tracks and cascades.
Tracks pass through the detector but do not deposit all their energy in the detector volume.
This means that the direction of the event can be reconstructed very well.
The signature left by such an event can be seen in figure \ref{fig:track}.
However, since the total energy is not deposited in the detector volume, the reconstruction of the energy is generally not optimal.
\begin{figure}
    \centering
    \includegraphics[width=10cm]{Plots/01_7_icecube/track.png}
    \caption{Tracklike event signature in the detectorvolume. The deposited energy is represented by the size of the spheres, the time of arrival by the colour (blue earlier, red later). The small glow of individual modules is unfiltered noise \cite{spiering}.}
    \label{fig:track}
\end{figure}
Cascades, on the other hand, deposit their entire energy in the detector volume, which means that the energy can be determined quite accurately.
Due to the shape of the cascade event, however, the original direction is very difficult to reconstruct.
Such an event can be seen in figure \ref{fig:cascade}.
Generally, the magnitude of the event's energy is the main statistical indicator of whether an event is astrophysical in origin, but the exact reconstruction of the event's direction is very important for a point source search.
Therefore, a dataset from tracks is usually used.
However, due to their high energy, cascade events can be used as a source catalogue \cite{steve_und_mirco}, but this is not the case in this thesis.
\begin{figure}
    \centering
    \includegraphics[width=10cm]{Plots/01_7_icecube/cascade.png}
    \caption{The filtered cascade event named \textit{Ernie} with a reconstructed energy of $\SI{1.04}{\peta\electronvolt}$. The deposited energy is represented by the size of the spheres, the time of arrival by the colour (blue earlier, red later) \cite{spiering}.}
    \label{fig:cascade}
\end{figure}

%theory
\chapter{Point Source Searches in IceCube} \label{sec:theory}

mention programs like skylab sky llh and yeah csky duh

maybe mention other methods than likelihood

Sensitivity and discovery potential correspond in principle to the number of signal events that must be injected, so that the resulting test statistic is $\SI{90}{\percent}$ above the median of the background statistic or $\SI{50}{\percent}$ above $\num{5}\sigma$ of the latter respectively.
These conditions can be seen schematically in figure \ref{fig:sens_disc_schem}.
Let the number of sufficient signal events then be $N_\text{sig}$, the sensitivity $sens$ and discovery potential $disc p$ then are
\begin{align}
  sens &= \frac{N_\text{sig,sens}}{A_\text{det}}E_0, \\
  disc p &= \frac{N_\text{sig,disc p}}{A_\text{det}}E_0,
\end{align}
with the total signal acceptance of the detector $A_\text{det}$ and the reference energy $E_0$.
The time-dependent sensitivity and discovery potential are given in a fluence rather than a flux by multiplying with $E_0^2$, these results can then be compared to other analyses.

\section{Likelihood in csky}

\begin{align}
  \CL(n_s,\Vmu)
  = \prod_i \Gb{
    \frac{n_s}{N} \CS(\Vx_i|\Vmu)
    + \Gp{1 - \frac{n_s}{N}} \CB(\Vx_i)
  }.
  \label{eq:lh}
\end{align}

%sources
\chapter{Event Topology and Datasets}
%general: cascades and tracks\\
what are the best sources and why\\
probability of being of astrophysical origin\\
History of EHE, HESE to bronze, gold alerts Thomas Kintschers gfu\\
what is a gfu\\
hängt ab von zen, dec, track\_len, bdt\_up (upgoing oder nicht) oder bei donwgoing: truncated energy, qtot (deponierte Ladung)\\
\_zen\_cut = np.radians(82.)\\
up: pass\_gold[m]   = pass\_precut[m] \& (qtot[m] > 10.\*\*np.polyval(\_coeff2, np.sin(dec[m])))\\
\_coeff2 = [  -4.06580, -10.60906,  -9.61048,   3.01219 ]\\
down: pass\_gold[m]   \= pass\_precut[m] \& (logTruncated[m] > \_cut2)\\
\_cut2   = 5.14\\

table of sources, skymaps\\

Historically events were categorized between high energy starting events (HESE) and extremely high energy events (EHE).
Starting events are as the name suggests events starting inside the detector volume (interaction vertex)

\cite{Aartsen_2017}

\section{Gamma-Ray-Followup}

%data
\section{Point Source Tracks Sample}
which dataset to use and why\\
description of the set: table of various info (livetime, number of events)\\
mention overlap in time due to testruns\\

\section{Monte Carlo Cleaning}
removing sources from data and mc\\
show sindec distribution of mc and filtered mc, maybe sindec-energy 2d hist\\
mention weighting (Powerlawflux)\\

%csky stuff
\chapter{Point Source Search using csky}

%tdepps stuff
\chapter{tdepps: An attempt at own time-dependent software}

%both results
\chapter{Conclusion and Outlook} \label{sec:outlook}
are the results measurable?\\
compare both analyses\\
compare with other analyses\\

what could have been done better, whats weird:\\
- mc gfu gold events about times 2: 21002 21220 are weird\\
- post trials should be done or all sources\\
- different gammas\\
- take gamma of gfu gold events\\
- spatial errors of sources could be accounted for\\
- take newest version of ps\_tracks\\
why no unblinding?\\
- takes too long\\

All in all, this work provides a good basis for getting started with point source searches using \textbf{csky} and can be used for further analyses by making minor changes.
For example, the data set can be enlarged or additional or different source locations can be investigated.

%outlook moved to previous chapter
%\chapter{Outlook}

%\input{content/02_struktur.tex}
%\input{content/03_grundlagen.tex}
%\input{content/04_figs_tabs.tex}

\appendix
% Hier beginnt der Anhang, nummeriert in lateinischen Buchstaben
%appendix
\chapter{Supplementary Material}

\section{Sources}

\begin{figure}
    \centering
    \includegraphics[width=\linewidth]{Plots/appendix/sources_energy.pdf}
    \label{fig:sources_energy}
    \caption{Histogram of the energy of the used sources seen in table \ref{tab:sources} in $\si{\tera\electronvolt}$.}
\end{figure}

\begin{table}
  \centering
  \caption{Table of the sources used in the time-dependent analysis. Additionally the signalness parameter is shown after which the sources were selected.}
  \begin{tabular}{ccrrc}
    \toprule
    Nr. & MJD &  $\delta$ in $\si{\degree}$ & $\alpha$ in $\si{\degree}$ & signalness in $\si{\percent}$ \\
    \toprule
      11 & 56819.20 & 11.45 & 110.65 & 99.70 \\ 12 & 56470.11 & 14.17 & 93.74 & 93.80 \\ 13 & 57951.82 & 25.16 & 208.39 & 86.60 \\ 16 & 58063.78 & 7.44 & 340.14 & 97.50 \\ 29 & 57340.87 & 12.71 & 76.16 & 95.70 \\ 37 & 55911.28 & 18.88 & 36.74 & 94.60 \\ 42 & 56226.60 & 27.91 & 169.80 & 92.60 \\ 43 & 56666.50 & 33.02 & 293.12 & 92.70 \\ 58 & 57478.57 & 15.48 & 151.22 & 85.10 \\ 63 & 56211.77 & -2.28 & 205.14 & 84.20 \\ 
    \toprule
    \label{tab:sources_time_dep}
  \end{tabular}
\end{table}


\backmatter
\printbibliography

\cleardoublepage
\input{content/eid_versicherung.tex}
\end{document}
